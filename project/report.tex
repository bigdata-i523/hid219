\documentclass[sigconf]{acmart}

\usepackage{graphicx}
\usepackage{hyperref}
\usepackage{todonotes}

\usepackage{endfloat}
\renewcommand{\efloatseparator}{\mbox{}} % no new page between figures

\usepackage{booktabs} % For formal tables

\settopmatter{printacmref=false} % Removes citation information below abstract
\renewcommand\footnotetextcopyrightpermission[1]{} % removes footnote with conference information in first column
\pagestyle{plain} % removes running headers

\newcommand{\TODO}[1]{\todo[inline]{#1}}

\begin{document}
\title{Unsupervised Learning For Detecting Fake Online Reviews}


\author{Syam Sundar Herle}
\affiliation{%
  \institution{Indiana University}
  \city{Bloomington} 
  \state{IN} 
  \postcode{47408}
  \country{USA}}
\email{syampara@iu.edu}


\begin{abstract}
Nowadays decision making done by organization and individuals are  
\end{abstract}

\keywords{HID 219, Opinionated Spamming}

\maketitle

\section{Introduction}
\subsection{The Need to Modernize Global Record Keeping}
Contracts, transactional records, and verification systems are part of the foundational core of the global economy. However, as Iansiti and Lakhani \cite{hbr} explain, these tools have not modernized to keep up with the needs of the rapidly evolving global economy and are ``like a rush-hour gridlock trapping a Formula 1 car.'' Records and transactions are still being managed as they were in the 20th century which creates broad consequences for nearly every industry including supply chain and healthcare.

In supply chain, data management methods for records and logistics are usually inconsistent across the different levels of a supply chain \cite{arbc4}. The outdated record management method encourages redundant data to be stored at the same organization as well as across the supply chain which increases IT maintenance costs and decreases trust and transparency \cite{arbc1}. These issues prevent a tertiary party, like the government, to effectively scrutinize records. 

Outdated data management processes also negatively impact healthcare. In the USA in 2014 healthcare fraud cost an estimated \$272 billion \cite{economist2014}, and in 2016, healthcare data breaches impacted over 27 million patients \cite{das2017}. Today, medical data management is stifled by antiquated technology that limits patients' ability to manage and control access to their electronic medical records \cite{ekblaw2016medrec}. 
In addition, pharmaceutical supply chains are enervated by current record-keeping technologies. Transactional records are rarely shared across pharmaceutical supply chain organizations which consequently increases inventory levels \cite{Nematollahi01}. As a result, total healthcare cost and the opportunity for counterfeit drugs increases \cite{Sahay01}. In addition, verification systems are often independent among supply chain retailers and prescribers. The lack coordination opens the door for ``doctor shopping'' and greater prescription medication abuse \cite{hitchingHealthcare}. 

\subsection{Rational Exuberance for Blockchain}
Blockchain, ``an open, distributed ledger that can record transactions between two parties efficiently and in a verifiable and permanent way'' \cite{hbr}, has the potential to resolve these and other fundamental problems of the global economy by overcoming many of the antiquated shortcomings of the traditional means of managing and verifying contracts and transactions. However, like TCP/IP in the 1970s and 1980s, blockchain is an immature technology that faces numerous challenges to mass adoption. In spite of its current limitations, blockchain is already seeing promising applications in various industries extending beyond just finance including healthcare and supply chain. One particularly exciting use case sits at the intersection of healthcare and supply chain for a more secure distribution system for opioid medications that could potentially mitigate the opioid crisis.

\section{Blockchain Overview}
\subsection{The Blockchain Framework}
Blockchain is a foundational technology comprised of numerous technological processes and entities. Some of the most significant pieces follow.

\subsubsection{Node} Nodes are the individual units connected to the blockchain network. They are computers with adequate software to maintain a blockchain. The blockchain network connects all the nodes and can read and write data to a block \cite{pabc1} \cite{pabc2}.

\subsubsection{Block} Blocks are the group of records, bundled together by nodes. They follow a specific set of rules and have limited size. Blocks are also linked to the last generated block, thus forming a chain \cite{pabc1}.

\subsubsection{Smart Contracts} Smart contracts are the codes with time stamps to represent a contract \cite{pabc1}. Iansiti and Lakhani \cite{hbr} believe that ``'smart contracts' may be the most transformative blockchain application at the moment,'' because they allow for automatic payments whenever contract conditions are met. 

\subsubsection{Submit Transaction} In case of a new transaction submission to the network, an individual node circulates it to all the other nodes in the network \cite{pabc1}. The main purpose of circulation is approval. 

\subsubsection{Transaction Approval} When a transaction is submitted and circulated in the network, each node verifies it. Invalid transactions are deleted \cite{pabc1}.

\subsubsection{Consensus} For multiple systems to work in a distributed network, they must have an agreement. Such a structure is useful in case of fault tolerance when those agreed set of protocols help to restore the data \cite{pabc1}.

\subsection{Data Types in Blockchain}
There are three major types of data stored on a blockchain, namely un-encrypted, encrypted and hashed \cite{arbc1}. 

\subsubsection{Un-encrypted Data} All the organizations have read access to the un-encrypted data. Such data is fully transparent and facilitates immediate dispute resolution \cite{arbc1}.

\subsubsection{Encrypted Data} The encrypted data can only be read by the organizations with the access to such data. This means an organization should have a decryption key to read to read the encrypted data. Encrypted data provides restricted access but is also stored in every node in the blockchain. In case of a dispute, the decryption key could be used by different organizations to rectify the entry or deletion of any record \cite{arbc1}.

\subsubsection{Hash Data} Hash data is also a hidden data, where hash keys act like fingerprints to represent changes or entry for any data record. Each organization can easily confirm their hash keys. Breaking the hash key is nearly impossible. Only the hash key is in the blockchain while the record data is stored off-chain by individual organizations. Data could be revealed, in case of a dispute, by the respective organization \cite{arbc1}.

\subsection{Benefits of Blockchain}
Blockchain's framework and data types provide such broad-ranging benefits that blockchain has been proposed as the ``cure'' to solve many of the world's problems. This exuberance stems from the fundamental benefits that are cornerstones to nearly every industry.

\subsubsection{Trust} Blockchains enable parties that do not know each other to trust each other. No single organization is trusted to maintain the records. Instead, all organizations must approve the contents of the record in order to avoid disputes. Therefore, records should have a time stamp and an origin proof. Normally, a third party facilitates this requirement. Blockchains can provide an alternate solution, where organizations jointly manage the records and preventing corruption by a single organization \cite{arbc1}. 

\subsubsection{Access} Blockchains allow for greater control over what information is and is not accessible. The technology enforces identical data to be stored by each organization. When one copy is updated, all the other copies are also updated. This eliminates the need for a third party to facilitate management of records \cite{arbc3}. Alternatively, different levels of read and write access could be provided to different organizations. Although some meta data should be stored in the public ledger. 

\subsubsection{Redundancy and Security} Blockchain also assists in providing security by disallowing redundancy at the same node. In the areas of logistics and inventory data, blockchain provides a new approach to supply chain management. The core logic of blockchain does not allow duplicate entries to be created in the same place \cite{arbc4}. A unique inventory can have a single entry with multiple updates, but not duplication. This prevents the organizations from creating false information. In the example of a drug inventory, the shipment status for a batch of drugs will be updated for everyone, everywhere. Each entry could be tracked back to its origin \cite{arbc4}. 

\subsubsection{Transparency} Transparency in a business helps to grow trust among organizations. Sharing information can improve relationships among these organizations. Without blockchain, transparency is hard to achieve. Blockchains can help improve the visibility of contracts, legal documents as well as other inter-organization data \cite{pabc1}. Organizations are not obligated to show all of their data. Some access can be provided to data that could be useful to other organizations and a shared collection of records can also be stored and managed by co-operation from different organizations.

\subsubsection{Low Transaction Costs} Through by-passing third-party verification systems such as brokers, lawyers, or banks, blockchain could significantly reduce transaction costs. Not only will this lower costs for existing transactions, it could open up the market for micro-payments \cite{hbr}. 

\subsection{Challenges to Blockchain Mass Adoption}
While it indeed has the potential to help a wide variety of the world's problems, it should not be viewed as a panacea.  Blockchain is not mature enough to support mass-market adoption and faces numerous challenges. Rabah \cite{rabah2017overview} states that to be effective, blockchain needs to overcome its shortcomings of lacking standard protocols, unclear regulation, large energy and computing power consumption, privacy, cultural adoption, and high initial capital requirements. Tapscott and Tapscott \cite{tapscott} agree that its current technical infrastructure is not sufficient, its energy consumption and computational requirements are not sustainable, and user-friendly systems have yet to be designed that would allow for mass market adoption.

Society would have to dismantle many technological, governance, organizational, and cultural barriers to create new foundations for a new world economy that relies heavily on blockchain \cite{hbr}. This will come at the cost of some existing societal norms, core business functions, and people's jobs \cite{hbr} \cite{rabah2017overview}. 

\subsection{Technology Adoption Lifecycle}
Iansiti and Lakhani \cite{hbr} argue that the process for mass adoption of blockchain may take longer than expected but will follow a fairly predictable technology adoption pattern that parallels the adoption of TCP/IP (transmission control protocol / internet protocol). TCP/IP started as \textit{single-use} and matured to \textit{localized uses}, \textit{substitutions}, and \textit{transformations}. It was introduced as a \textit{single-use} in 1972 for e-mail in ARPAnet, a precursor to commercial internet for the US Department of Defense. Met with skepticism, this technology slowly gained traction among some firms in the 1980s and early 1990s for \textit{localized use} and did not become mainstream until the emergence of World Wide Web in the mid-1990s. This then paved the road for infrastructure companies to provide the necessary hardware and software to establish ``plumbing'' systems for the internet. Once the technical infrastructure was mature enough, companies then developed businesses that \textit{substituted} existing services with online services (such as Amazon books instead of Borders). Finally, a wave of companies created \textit{transformative} applications that fundamentally changed service experiences (such as Napster in the music industry or Skype in telecommunications).

Similarly, blockchain was also launched for a \textit{single use} in 2009 for Bitcoin, a virtual currency. Blockchain has matured to extend beyond cryptocurrencies and is now being applied for various \textit{localized uses} including in healthcare and supply chain. It took over 30 years for TCP/IP to realize its potential, and blockchain will likewise require decades to mature into a revolutionary economic force. However, companies can start planning for this revolution today and implement blockchains that follow seven design principles \cite{hbr} \cite{tapscott}.

\subsection{Seven Design Principles for Blockchain}
Tapscott and Tapscott \cite{tapscott} in their book \textit{Blockchain Revolution} propose seven design principles that, when appropriately applied, can help blockchain move down the technology adoption lifecycle and create more honest, cost-effective, and accountable systems.

\subsubsection{Networked integrity} Because all organizations on the blockchain must approve updates, ``Participants can exchange value directly with the expectation that the other party will act with integrity.'' \cite{tapscott}. 

\subsubsection{Distributed Power} Since the blockchain is distributed across a broad network, it cannot be dismantled by authoritarian power, hackers, or other bad actors. There are no single points of failure and the blockchain can still perpetuate even if numerous nodes are compromised \cite{tapscott}.

\subsubsection{Value as Incentive} Blockchains can align incentives of individual participants with the interests of the entire blockchain. This minimizes organization problems and conflicts of interests \cite{tapscott}.

\subsubsection{Security} Blockchains can protect against hackers, malware, ransomware, and identity theft by using a variety of security features. Public key infrastructures, private keys, public keys, and verification methods verify participant activities and prevent bad actors from overriding the network \cite{tapscott}. 

\subsubsection{Privacy} Blockchains can and should provide participants with the freedom to expose as little or as much information about themselves as they desire. This allows a participant to act anonymously when desired or to share sensitive information with only appropriate parties when needed \cite{tapscott}.

\subsubsection{Rights Preserved} To protect against counterfeit items, a blockchain can serve as a public ledger of ownership \cite{tapscott}.

\subsubsection{Inclusion} Currently, access to certain financial services is limited to those who are deemed ``creditworthy''. Blockchains can and should have significantly lower bars of entry that are not managed by banking institutions so that even a poor rural former on a remote corner of Earth who isn't creditworthy, could participate in the blockchain \cite{tapscott}.

\section{Blockchain Applications}
\subsection{Supply Chain}
Blockchain, being a public ledger, can be used in different domains with slight variation in its core attributes. While the general implementation says that the data of a single block is public to all the nodes, different sets of access rights could be provided to different classes of users.  Such implementation of blockchain could be applied to a supply chain network. 

A supply chain requires the involvement of various parties helping each other. This is generally a one-to-one chain network. Often, each organization uses different technologies for record keeping. Record keeping could involve any information ranging from direct communications to logistics. Trust is an important issue between organizations. Most of the organizations in a supply chain keep individual records, which are not public to other organizations in the supply chain. Organizations share some information like contracts or notarized data. An efficient management of such shared data can be accomplished using a blockchain. The blockchain provides the ability to collect, record, and notarize different types of shared data \cite{arbc1}. 

Blockchain could also facilitate storing and maintaining logistics data. Such an application could be useful in the field of healthcare, where the government wants to monitor the supply of drugs. An ideal scenario for this would be to mitigate issues like the opioid crisis. Blockchain technology could simplify storage and management of trusted information. It could provide easy access of such critical public sector information to government organizations while providing data security \cite{arbc2}. Blocks comprise of the data records. When these blocks are added to the chain, they become immutable. This means they cannot be deleted or changed by a single organization \cite{arbc2}. A consensus has to be reached by a majority of the organizations for changing any record. Such a feature helps to maintain the security of the records by eliminating data corruption. Each block is verified and managed using some shared protocols. This process can be automated to allow ease of data entry. Two uses cases are for counterfeit detection and data analysis.

\subsection{Healthcare}
Representing over 17\% of the United States' GDP, healthcare costs continue to soar \cite{hitchingHealthcare}. More effective data management could address many of healthcare's fundamental issues, and according to a 2011 McKinsey report \cite{mckinsey2011}, more effective health data management could save \$300 billion annually. Current innovations focus on placing patients at the center, privacy and access, completeness of information, and cost \cite{hitchingHealthcare}. Three interesting applications of blockchain for healthcare are in claims adjudication, cyber security and healthcare IoT, and electronic medical records \cite{das2017}.

\subsubsection{Claims Adjudication and Fraud Prevention}
In 2014, the Economist estimated that the United States wasted \$272 billion dollars on healthcare fraud \cite{economist2014}. Blockchain could not only minimize fraudulent billing; but, by automating claims adjudication and billing processes, obviate the need for administrative and transactional costs through third parties. Gem Health and Capital One are developing a blockchain-based solution for healthcare claims management \cite{das2017}.

\subsubsection{Cyber Security and Healthcare IoT}
In 2016, there were 450 reported health data breaches, impacting 27 million patients. Hacking and ransomware were responsible for 27\% of these breaches. Each additional connected medical device serves as a potential entry point for bad actors. With an estimated 20-30 billion healthcare IoT devices by 2020, blockchain could secure these devices and protect confidential data. Telstra, IBM, and Tierion are three companies that are developing cyber security solutions for connected healthcare devices \cite{das2017}.

\subsubsection{Electronic Medical Records} Beleaguered by stifled technology development, limited ownership control by patients, fragmented information systems, and risks of electronic protected health information hacking, electronic medical records have perhaps the most important use cases for blockchain \cite{yuan2016blockchains}. Blockchain can provide interoperability of healthcare information, improved security, patient-centric control, and immutable records \cite{das2017}. Three examples of blockchain-based EMRs include MedRec, Medicalchain, and the Estonian eHealth Foundation. First, by leveraging smart contracts on the Ethereum blockchain, MedRec is a prototype system that provides patients with ``one-stop-shop access to their medical history'' and shows promise to give ownership of health information back to the patients who can selectively share access through a modern API interface in a secure manner \cite{ekblaw2016medrec}. Second, Medicalchain is a permissioned blockchain distributed on networks of international healthcare providers that allow patients to transfer medical records across national borders \cite{hitchingHealthcare}. Third, a data security company called Guardtime is using its Keyless Signature Infrastructure system in partnership with the Estonian eHealth Foundation to store Estonian health records on a blockchain.

\section{The Opioid Crisis}
\subsection{Addiction Risk}
Since the late 1990s, pharmaceutical companies have downplayed the addictive risk of opioids \cite{opsis1}. However, the addictive nature of prescribed opioid painkillers increases the ``potential for unforeseen adverse events for the patient, including overdose, experience of physiological dependence and subsequent withdrawal, addiction, and negative impacts on functioning'' \cite{Vowles01}. Patients with wholesome medical intentions often fall victim to the pills' addictive nature. Misuse and eventual abuse of prescribed opioid painkillers are common: 21\%-29\% of patients prescribed opioids for chronic pain misuse them while 7.8\%-11.7\% develop an addiction \cite{Vowles01}. Moreover, an opioid addiction often serves as a gateway to other illegal drug use. With similar highs, prescription opioid addicts often transition to heroin, an illicit street-made opioid, since it is cheaper and easier to obtain. In fact, 4\%-6\% of patients using prescribed opioids develop a heroin addiction \cite{opsis1}. Whereas, 75\% of heroin users began their opioid addiction with prescription opioids \cite{Cicero01}.

Despite these risks, opioids are still prescribed at alarming rates. In fact, the United States, with about 5\% of the world's population, consumed 80\% of the world's opioid prescriptions from 2001-2010 \cite{Vowles01}. Between 1999 and 2015 the amount of prescribed opioids painkillers such as codeine, fentanyl, oxycodone, Demerol, and Vicodin quadrupled. In the same time period, opioid-related deaths also quadrupled.
 
\subsection{Health Impact}
The epidemic has become so severe that in October 2017 President Trump was forced to declare it ``a national health emergency'' \cite{opsis3}.    With no signs of stopping, this epidemic is burgeoning across America killing nearly 91 people a day \cite{opsis10}.
 
In 2015, 33,091 Americans died from an opioid overdose with rural white males at the greatest risk of an opioid overdose.  White Americans (27,056) died the most, followed by black Americans (2,741), and Hispanic American (2,507). Generally the middle-aged population was most at risk with the following percent distributions by age group \cite{opsis4}:
\begin{itemize}
\item Aged 0-24: 10\% of the opioid-related deaths
\item 25-34: 26\% 
\item 35-44: 23\% 
\item 45-54: 23\% 
\item 55+: 19\% 
\end{itemize}
Males die nearly twice as frequently from an opioid overdose, representing 65\% deaths compared with 35\% for females \cite{opsis4}.

\subsection{Financial Impact} 
The health impacts are the primary reason for concern, but the financial liability associated with the epidemic is also increasing. The estimated financial impact of the crisis grew from \$55.7 billion in 2007 \cite{Birnbaum01} to \$78.5 billion in 2013 \cite{Florence01}. Of the total economic burden, roughly 25\% or \$20 billion is conveyed to the public sector \cite{Florence01}. Partitioned between workplace, healthcare, and criminal justice costs, the overall financial burden will continue to rise until a reversal in current trends. 

Opioid drug makers are also exposed to significant financial and legal liabilities as lawsuits accusing pharmaceutical companies of deceptive marketing are commonplace. After a U.S. Justice Department probe in 2007, the maker of OxyContin pleaded guilty to federal charges and paid \$634.5 million. In later cases, OxyContin maker Purdue Pharma LP settled two additional cases for a combined \$43.5 million. Since then governments litigating the culpability of opioid drug makers include ``South Carolina, Oklahoma, Mississippi, Ohio, Missouri and New Hampshire as well as cities and counties in California, Illinois, Ohio, Oregon, Tennessee and New York'' \cite{opsis11}. In a suit filed in April 2017 against the three largest drug retailers in the USA - CVS, Walgreens, and Walmart - lawyers for plaintiffs Cherokee Nation claim that the ``Defendants turned a blind eye to the problem of opioid diversion and profited from the sale of prescription opioids to the citizens of the Cherokee Nation in quantities that far exceeded the number of prescriptions that could reasonably have been used for legitimate medical purposes'' \cite{opsis5}.

\subsection{Responses to Mitigate the Crisis}
The private sector, government, and academia alike recognize the importance of solving this crisis and are implementing strategies to help mitigate the opioid crisis. 

\subsubsection{Private Sector}
Drug retailers are taking immediate action. In September 2017, CVS pharmacy announced actions to limit patient supply of prescription opioids to seven days, to restrict the strength of opioids dispensed for first time patients and to install 750 more in-store drug disposal kiosks \cite{Charles01} \cite{Hansen01}.

A longer-term private sector solution is through the use of radio frequency identification (RFID) technology as a method to improve supply chain security \cite{Taylor01} \cite{Wyld01}. RFID tracking tags are small microchips that are either printed, etched, stamped, or vapor-deposited onto product labels and are intended to replace barcodes. RFID can be read without direct line of sight and at distances up to 30 feet. Research shows that RFID tags have the potential to reduce costs, increase transparency, and identify counterfeit lots. RFID tags have many advantages over current barcode tracking methods. RFID tags can hold up to 32,000 alphanumeric characters compared to just 20 in a barcode. RFID tags have a much higher upfront cost but decrease total supply chain cost due to the timely process to scan each individual barcode. And unlike RFID tags, barcodes are susceptible to wear and tear and are easily replicated. RFID technology also has its flaws. In addition to the higher upfront cost, each tag costs between 5-10 US cents, signficantly higher than bar codes. Moreover, they are vulnerable to electromagnetic interference and poor manufacturing, are larger, and require a much larger IT infrastructure \cite{Taylor01} \cite{opsis9}. From a security and transparency perspective, RFID technology is a good option to conform to The Drug Quality and Security Act \cite{DQASA}.

\subsubsection{Government}
Through policy and politics, the federal government is attempting to find solutions to the epidemic. In the same address President Trump declared the opioid epidemic a national health crisis, he proposed ``really tough, really big, really great advertising'' \cite{opsis6}. Tom Price of the U.S. Department of Health and Human Services outlined a more detailed federal long-term plan including, ``improving access to treatment and recovery services, promoting use of overdose-reversing drugs, strengthening our understanding of the epidemic through better public health surveillance, providing support for cutting edge research on pain and addiction, and advancing better practices for pain management''\cite{opsis7}. Additionally, President Trump's Commission on Combating Drug Addiction and the Opioid Crisis repeatedly mentions ``data sharing'' as a method to cope and limit the opioid crisis \cite{opsis3}.

Multiple studies indicate that states with strong prescription drug monitoring programs (PDMPs) show a significant reduction in the number of opioid-related deaths \cite{pardo01} \cite{patrick01}. Evidence suggests that 72\% of physicians were aware of their states' PDMPs in 2015, but only 52\% used their services. Physicians noted difficulties understanding the data formats and retrieval systems as the main barriers to continual use of PDMPs \cite{Rutkow01}. As a result, low registration rates are common in the 49 states that offer some form PDMPs \cite{Hawk01}.

Increasing access to Naloxone, an opioid antagonist that rapidly reverses the opioid overdose damage, may be the most important immediate solution to reducing opioid-related deaths \cite{Hawk01}. Between 1998 and 2014, 52,283 naloxone kits were distributed among the 30 states with naloxone distribution programs resulting in 26,453 overdose reversals \cite{Hawk01}. 27 states have "third-party prescription" laws that allow physicians to prescribe Naloxone to family and friends of individuals with an opioid addiction \cite{Hawk01}. To further reduce opioid-related deaths states must reduce malpractice liability for physicians prescribing Naloxone and make Naloxone available without a prescription \cite{Hawk01}.

In addition, states have started to pass legislation protecting Good Samaritans. As of 2014, 23 states had laws protecting cooperating bystanders, from low-level misdemeanors and drug possession. Without these laws, bystanders are subject to criminal charges and even murder if it is proven they supplied the deadly drugs. Consequently, these laws are necessary to encourage immediate life-saving calls to 911 \cite{Burris01} \cite{Hawk01}. 

Other solutions states should consider is access to medical marijuana, as Pardo \cite{pardo01} found that states with legal medical marijuana dispensaries have lower opioid-related deaths.

\subsubsection{Academia}
Academic research is helping to propose effective solutions to the opioid crisis. For example, Indiana University announced plans to commit \$50 million and 70 researchers to find solutions that lead to a decline in opioid-related deaths \cite{Rudavsky01}. In a similar proposal, researchers at the Network for Public Health Law, Boston University, and Northeastern University proposed a four-step solution including ``improving clinical decision making and access to evidence-based treatment, investing in comprehensive public health approaches, and re-focusing law enforcement response ''\cite{Davis01}.

\section{An Overview of Pharmaceutical Supply Chains}
\subsection{Network Nodes}
Forward facing supply chain activities occur before a customer purchase. In a pharmaceutical supply chain, forward facing nodes includes manufacturers, warehouses, distributors, and retailers. Reverse facing supply chain activities occur after the sale and include collecting, recycling, redistributing, and disposing of unwanted medications.  
\subsubsection{Primary Manufacturing} Produces the main active ingredient \cite{Shah01}.
\subsubsection{Secondary Manufacturing} Often at a different geographic location for tax and labor reasons, secondary manufacturers combine the active ingredients produced by primary manufacturers and adding excipient substances. Secondary manufacturers produce distribution ready SKU medications through one or more of the following processes: granulation, compression, coating, quality control, and packaging \cite{Shah01}.
\subsubsection{Market Warehouses and Distribution Centers} Due to the cost of setup and cleaning, it is common for primary manufacturers to produce a years' worth of active ingredients for a particular medication in one batch. This strategy creates a lot of excess finished and work-in-progress inventory that is stored in warehouse and distribution centers \cite{Shah01}.
\subsubsection{Wholesalers} Roughly 80\% of demand flows through wholesalers. The industry is highly competitive and consolidated. The largest five wholesalers accounted for roughly 45\% of industry revenue \cite{Shah01} \cite{Hoovers01}.
\subsubsection{Pharmacies and Hospitals} The last node on the pharmaceutical supply chain before medications are distributed at a patient level. Major retailers include pharmacies CVS, Walgreens, Walmart, and Rite Aid and hospital systems such as Community Health Systems, Hospital Corporation of America, and Ascension Health \cite{Shah01}. 
\subsubsection{Reverse Supply Chain} The reverse supply chain is often overlooked as a key component of the pharmaceutical supply chain network. Few people take their unwanted medications to proper collection sites. Instead, medications are discarded in the trash and sewage. In fact, in 2003 the world disposed of at least \$760 million worth of prescription medications \cite{Hua01}. By 2014, this number ballooned to an estimated \$5 billion \cite{Lenzerg01}. The roughly 10 million unused and unexpired prescription medications could be recycled and reused, but instead improper disposal leads to dangerous compounds in sewage effluent, surface water, and even drinking water \cite{Hua01} \cite{Lenzerg01}.  Hua, Tang, and Wu \cite{Hua01} suggest a combination of government subsidies, penalties, and marketing to encourage drug makers to collect unwanted and expired medications. 

\subsection{Weaknesses}
The nature of the current pharmaceutical production and supply chain system creates multiple weaknesses. 
\subsubsection{Lead Time} Lead times, the time it takes between manufacturing and end sale, can take up to 300 days \cite{Shah01}. As a result, high safety stocks are needed to react to future demand.
\subsubsection{High Service Levels} The necessity for on-time pharmaceutical products forces retailers to maintain high service levels, the targeted rate of stock-outs. In many cases and especially in hospitals, patient health relies on having the right medication at the right time. A failure to meet this immediate demand could lead to fatal consequences \cite{Kelle01} \cite{Hua01}.
\subsubsection{Imbalance of Information} Another major disadvantage is the lack of collaboration between raw material suppliers, manufacturers, warehouses, wholesalers, and retailers. ``The problem is that the different decision-makers do not have access to the same information regarding the state of the entire supply chain network, and in addition they usually operate under different objective functions'' \cite{Sahay01}. In this decentralized method, manufacturers have a difficult time forecasting demand. In addition, an imbalance of information between supply chain nodes increases cost and stock-outs. However, Nematollahi, Hosseini-Motlagh, and Heydari \cite{Nematollahi01} found that collaborative decision making through information sharing can increase economic benefits for the entire supply chain while also increasing drug fill rate.
\subsubsection{Manufacturing Strategy} The mixture of manufacturers `push' strategy and retailers `pull' strategy, results in high safety stocks. At any given point, there is usually 4 to 24 weeks of finished goods that has yet to be delivered to patients \cite{Shah01}. 
\subsubsection{Large Network} Medications pass through several nodes before they are delivered to the market. Safety and security issues face organization conflicts as the capital cost to prevent theft and mismanagement is not equally spread across the supply chain. The number of nodes also increases the likelihood for counterfeits to enter the market. Between each node, medications are shipped and handled between multiple parties and often times across national and state borders \cite{Shah01}.
\subsubsection{Government Regulation} The government heavily regulates pharmaceutical supply chains to ensure a safe and steady supply of medications. The Drug Quality and Security Act \cite{DQASA} signed by President Barack Obama in 2013 introduced new regulations for the manufacturing and the distribution of pharmaceutical products. The policy mandates the creation of systems to trace lot-level transactions and systems to verify product legitimacy. In addition, any company within the supply chain must obtain federal licensure and authenticate the licensure of their trading partners. These required changes place immense financial pressure on pharmaceutical companies, drug distributors, and prescribers to develop sustainable supply chain solutions. The 2023 deadline gives pharmaceutical companies time to test and implement the most sustainable and practical solution \cite{opsis8}.
\subsubsection{Counterfeits} High inventory levels increase supply chain cost, the potential for theft, and the introduction of counterfeits. It is estimated that 10\% of the worldwide pharmaceuticals are counterfeit and approaching 25\% in developing countries \cite{Kelesidis01}. Pharmaceutical companies lose an estimated \$200 billion annually due to counterfeit drugs \cite{das2017}.  

\section{Blockchain's Potential to Mitigate the Opioid Crisis}
\subsection{Moving Opioid Distribution onto the Blockchain}
Blockchain can mitigate the opioid crisis through more secure opioid distribution. The 2013 federal passing of The Drug Quality and Security Act \cite{DQASA} provides pharmaceutical supply chain organizations with the necessary regulatory incentives to quickly move onto the blockchain.

The first step to moving opioid distribution onto the blockchain rests in the initial infrastructure investment plan for development and maintenance. The next step is to establish the policies and security clearances of each organization \cite{Christidis01}. Once these critical questions are answered, an opioid distribution blockchain would be similar to blockchains in other industries. Each blockchain would start with the genesis node created by the raw material supplier. From there on, each additional downstream node would timestamp an additional hash. When the opioid eventually reaches the patient, the block would contain information on all involved supply chain nodes with timestamps and distribution information including prescribing physician and pharmacist. 

The Hyperledger design principles \cite{Cocco01} \cite{Hyperledger01}, Tapscott and Tapscott's seven design principles for blockcahin \cite{tapscott} and BlockSci \cite{Kalodner01} analysis protocols should be included in the design of the blockchain.

\subsection{Benefits}
\subsubsection{Cost Savings} As a proactive cost saving maneuver, drug makers and retailers can move onto the supply chain to prevent future litigation \cite{Noguchi01}. In addition, blockchain automation saves time and operating costs \cite{Hyperledger01}.  
\subsubsection{Reducing Lead Times} Collaborative record-sharing is the foundation and ultimate strength of blockchain technology. Nematollahi, Hosseini-Motlagh, and Heydari \cite{Nematollahi01} show that collaborative record-sharing among pharmaceutical nodes increases both the social and economic effectiveness of the supply chain. The economic benefits realized through the reduction of the total supply chain inventory levels also decreases lead times. 
\subsubsection{Collaborative Information Sharing} Blockchain technology has the potential to reduce the opioid epidemic through transparent and decentralized record keeping. In particular, blockchain has the potential to identify prescription drug fraud. Currently without blockchain, opioid addicts can take advantage of the incomplete feedback between physicians and pharmacists by ``doctor shopping'', modifying, and duplicating prescriptions \cite{hitchingHealthcare}. With pharmaceutical records on the blockchain, this type of activity is easily identifiable. 

Blockchain can reduce illegal opioid prescribing and distribution. In the current centralized record keeping system, the U.S. Drug Enforcement Administration (DEA) relies the Controlled Substances Act of 1970, that requires drug companies to report unusually large or otherwise suspicious orders \cite{Higham01}. Drug makers on the other hand claim their responsibility to report is too vague.  As a result, identifying ``pill mills'' is unnecessarily difficult and time consuming. The DEA's pharmaceutical unit has 600 investigators \cite{Higham01}. With blockchain, record keeping is standardized and accessible to all parties with the correct cryptographic keys. 

\subsubsection{Post-Sale Opioid Collection} Blockchain technology can also increase the usefulness of post-sale opioid collection. Current medication packaging lacks 2D DataMatrix barcodes making it nearly impossible to identify historical information such as who is returning their medication, who prescribed and sold the medication, and when the medication was prescribed and returned \cite{Walles01}. Blockchain can trace this information leading to better post-sale analysis. In turn, this information can be studied to improve prescribing methodology.
\subsubsection{Counterfeit Detection} Blockchain can reduce the high prevalence of illicit counterfeit drugs. Blocks are immutable, that is once a block is created it cannot be deleted or erased \cite{hitchingHealthcare}. In addition, each batch of product can be traced back to its origin. This means that each batch will have a block of code associated with it. If a batch does not have its presence in the blockchain, then it can be deemed as a counterfeit \cite{arbc2}. Furthermore, blocks with abnormal distribution patterns can be flagged and removed from the supply chain. Creating illicit blocks is easily identifiable as all new blocks must be approved by all parties on the blockchain. 
\subsubsection{Data Analysis} Academic institutions and researchers should have access to superkeys to analyze trend analysis to provide predictions and usage patterns over various locations and times of the year \cite{arbc5}. Data analysis can provide both a descriptive and predictive overview of the opioid supply chain. 

\subsection{Outlook}
Moving forward, blockchain must overcome multiple adoption risks. Blockchain monopolization in the pharmaceutical supply chain will reduce the effectiveness, safety, and security of the system. Future governmental regulations can prevent mergers and acquisitions in this industry. In addition, future quantum computing power may be strong enough to break cryptographic keys. Investing in security is necessary for future blockchain success. Lost keys will result in irretrievable data; thus, developing a system to overcome this issue is critical. Lastly, blockchain technology is only as good as its users. Encouraging accurate and timely data entry will ultimately define the usability of blockchain in prescription drug distribution \cite{hitchingHealthcare}.

Nonetheless, blockchain can play a role in mitigating the deadly opioid epidemic by providing cost savings, reducing lead times, facilitating information sharing, facilitating post-sale opioid collection, detecting counterfeit, and providing rich data for analysis.

\section{CONCLUSION}
Although still in its infancy, blockchain has the potential to be just as transformative as TCP/IP. Early and potential applications in healthcare and supply chain suggest that blockchain is indeed moving along the path of technology adoption. Because blockchain is a low-cost solution for supply chain management and provides security and transparency, it can be used for digital data and communication to overall the distribution of controlled substances such as opioids but this model has yet to be tested. But the computation and infrastructure cost for the entire model is low and should be tested to develop a proof of concept system that leverages blockchain to more securely distribute prescription opioids. A prototype model of blockchain can be developed which emulates the current structure of a pharmaceutical supply chain. Such a model can be vital to test out the flaws of blockchain and how to accurately tailor it to the specific use case.  

\begin{acks}
The authors would like to thank Dr. Gregor von Laszewski for his support and suggestions to write this paper.
\end{acks}

\bibliographystyle{ACM-Reference-Format}
\bibliography{report} 

\appendix

\section{Issues}

\DONE{Example of done item: Once you fix an item, change TODO to DONE}

\subsection{Assignment Submission Issues}

    \TODO{Do not make changes to your paper during grading, when your repository should be frozen.}

\subsection{Uncaught Bibliography Errors}

    \TODO{Missing bibliography file generated by JabRef}
    \TODO{Bibtex labels cannot have any spaces, \_ or \& in it}
    \TODO{Citations in text showing as [?]: this means either your report.bib is not up-to-date or there is a spelling error in the label of the item you want to cite, either in report.bib or in report.tex}

\subsection{Formatting}

    \TODO{Incorrect number of keywords or HID and i523 not included in the keywords}
    \TODO{Other formatting issues}

\subsection{Writing Errors}

    \TODO{Errors in title, e.g. capitalization}
    \TODO{Spelling errors}
    \TODO{Are you using {\em a} and {\em the} properly?}
    \TODO{Do not use phrases such as {\em shown in the Figure below}. Instead, use {\em as shown in Figure 3}, when referring to the 3rd figure}
    \TODO{Do not use the word {\em I} instead use {\em we} even if you are the sole author}
    \TODO{Do not use the phrase {\em In this paper/report we show} instead use {\em We show}. It is not important if this is a paper or a report and does not need to be mentioned}
    \TODO{If you want to say {\em and} do not use {\em \&} but use the word {\em and}}
    \TODO{Use a space after . , : }
    \TODO{When using a section command, the section title is not written in all-caps as format does this for you}\begin{verbatim}\section{Introduction} and NOT \section{INTRODUCTION} \end{verbatim}

\subsection{Citation Issues and Plagiarism}

    \TODO{It is your responsibility to make sure no plagiarism occurs. The instructions and resources were given in the class}
    \TODO{Claims made without citations provided}
    \TODO{Need to paraphrase long quotations (whole sentences or longer)}
    \TODO{Need to quote directly cited material}

\subsection{Character Errors}

    \TODO{Erroneous use of quotation marks, i.e. use ``quotes'' , instead of " "}
    \TODO{To emphasize a word, use {\em emphasize} and not ``quote''}
    \TODO{When using the characters \& \# \% \_  put a backslash before them so that they show up correctly}
    \TODO{Pasting and copying from the Web often results in non-ASCII characters to be used in your text, please remove them and replace accordingly. This is the case for quotes, dashes and all the other special characters.}
    \TODO{If you see a figure and not a figure in text you copied from a text that has the fi combined as a single character}

\subsection{Structural Issues}

    \TODO{Acknowledgement section missing}
    \TODO{Incorrect README file}
    \TODO{In case of a class and if you do a multi-author paper, you need to add an appendix describing who did what in the paper}
    \TODO{The paper has less than 2 pages of text, i.e. excluding images, tables and figures}
    \TODO{The paper has more than 6 pages of text, i.e. excluding images, tables and figures}
    \TODO{Do not artificially inflate your paper if you are below the page limit}

\subsection{Details about the Figures and Tables}

    \TODO{Capitalization errors in referring to captions, e.g. Figure 1, Table 2}
    \TODO{Do use {\em label} and {\em ref} to automatically create figure numbers}
    \TODO{Wrong placement of figure caption. They should be on the bottom of the figure}
    \TODO{Wrong placement of table caption. They should be on the top of the table}
    \TODO{Images submitted incorrectly. They should be in native format, e.g. .graffle, .pptx, .png, .jpg}
    \TODO{Do not submit eps images. Instead, convert them to PDF}

    \TODO{The image files must be in a single directory named "images"}
    \TODO{In case there is a powerpoint in the submission, the image must be exported as PDF}
    \TODO{Make the figures large enough so we can read the details. If needed make the figure over two columns}
    \TODO{Do not worry about the figure placement if they are at a different location than you think. Figures are allowed to float. For this class, you should place all figures at the end of the report.}
    \TODO{In case you copied a figure from another paper you need to ask for copyright permission. In case of a class paper, you must include a reference to the original in the caption}
    \TODO{Remove any figure that is not referred to explicitly in the text (As shown in Figure ..)}
    \TODO{Do not use textwidth as a parameter for includegraphics}
    \TODO{Figures should be reasonably sized and often you just need to
  add columnwidth} e.g. \begin{verbatim}/includegraphics[width=\columnwidth]{images/myimage.pdf}\end{verbatim}

re

\end{document}
