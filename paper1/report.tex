\documentclass[sigconf]{acmart}

\settopmatter{printacmref=false}
\renewcommand\footnotetextcopyrightpermission[1]{}
\pagestyle{plain}

\usepackage{dirtytalk}

\begin{document}
\title{Big Data and Transportation}


\author{Syam Sundar Herle}
\orcid{HID219}
\affiliation{%
  \institution{Indiana University}
  \streetaddress{711 N Park Ave}
  \city{Bloomington} 
  \state{Indiana} 
  \postcode{47408}
}
\email{syampara@iu.edu}

\begin{abstract}

With the rise of population in cities, commutation by road or rail have become hard for people. Transit plays major role in public and private day to day life, but there are limited number of system to address the transportation issues. On other hand Big Data have proved to be more effective and helpful in most of the sectors and business. Big data is all about realizing full potential of large data set by acquiring, storing and managing by advanced technologies and optimizing techniques. With the help of new era of technologies like internet, social media, traffic camera,feeds and smart-phones one can have access to more real-time data which contains abundant of information, which can be used in transportation sector. We can use advance Big Data technologies like Spark on those real-time data set to address transportation issues to build next level Intelligent Transportation Systems (ITS). 

\end{abstract}

\keywords{i523, HID219, Big Data, Intelligent Transportation Systems}

\maketitle

\section{Introduction}

In modern era, technologies plays a major role in our day to day life changing the way of our interaction and livelihood. Smart-phones have become unsaid finger for our hand and internet have pushed communication to next level. The revolution made by technologies and internet have resulted in collection of large real-time data. Real-time data coming from transport camera's, mobile devices, social media's and other sources need to be dealt to provide easy decision making system for commuter in order to set a effective transportation system. According to \cite{} \say{the volume and speed at which data are generated, processed and stored is unprecedented}. Said that, Big data is all about realizing full potential and revealing hidden patterns of large data set by acquiring, storing and managing by advanced technologies and optimizing techniques. 

The overwhelming real-time data collected from mobile devices and social media like Facebook, Twitter while travelling contains not only the location based information of the users but also the nature of traffic in hidden manner. These real-time data collected from social media applications and traffic camera can be used to predict the nature of road ahead, optimized route, traffic forecasting and predict them before they occur for the public by Intelligent Transportation system(ITS)\cite{}. The nature of the data are in diverse manner ranging from structured to unstructured, as these are in real-time the consumed data size would be in terabytes. Because of the huge amount of big data and its nature there is a urgent need of addressing the issues like storing, managing and analyzing which are beyond the capacity of the traditional tools which are used for analysis.

\section{Transportation}

Transportation as a means of transporting goods and people between different location, is a vital element of modern society. Since the earliest days of the industrial revolution, transportation has facilitated economic development by moving materials, resource, products and people \cite{}. Transportation around the world have some common problems, which causes disruption of not only road traffic but also it has effects on the economic and as well as ecosystem of country.

Common problems in transportation,

\begin{itemize}
\item Road traffic
\item Environmental Pollution
\item Accidents
\item Inability of forecasting traffic
\item Road Maintenance
\item Road Construction/Maintenance 
\end{itemize}

Creation of a transportation system need to address the above problems, when implementing a transportation, government needs to address to avoid the future traffic congestion in case of road maintenance of road construction. Road accidents costs more than life cost, like insurance, rehabilitation cost, property damage cost and so on. Pollution impact also need to be addressed when designing and implementing a effective transport system, if CO2 emission is unchecked it may lead to health related problem and global warming.

Apart from mentioned problem's, creation of a transport system also needs to address some to the issues related to city planning also, the problems can be categorized in the following subsections,

\begin{enumerate}
\item City Plan 

The plan of the city is very crucial, specifically in cities as the traffic congestion as to be addressed. 

\item Police and Law Enforcement 

Consideration of traffic law is also very important, speed limits are also important to taken into consideration.

\item Event Gathering 

During specific social event gathering, road block and alternative route have to be taken into consideration.

\end{enumerate}

The above issues highlights specifically the situation of the road congestion and road traffic, so the government or transport system planning body should take necessary considerations in the above stated problems by investing sufficient time and money for installing the surveillance and tools. 


\section{Big Data}

As defined earlier, Big data is all about realizing full potential and revealing hidden patterns of large data set by acquiring, storing and managing by advanced technologies and optimizing techniques. According to \cite{} Big Data has also been defined as by the four V's:

\begin{enumerate}
\item Volume

The mass of amount of the data, which indicates to the data in low level. Lot's of data collected from social media platform are variety and huge amount in nature, which consists of structured and unstructured. Looking up the unstructured data will be futile and we cannot mine any useful information from the raw unstructured data. Advance technologies like Hadoop, Map Reduce needs to employed to manage storage and mine the unstructured data.

\item Velocity

This can be defined by the rate of pace at which the data are accumulated. In some real-time applications like Internet of Things (IoT) related to health care systems and transportation systems there is a need to evaluate real-time data. Instead of writing high streams of data we need to store it directly to memory to evaluate them. 
\item Variety

Usually variety can be defined as mixture, when coming to data there will be structured, semi-structured and unstructured. Data like text, audio and video will be unstructured in nature, we need to process to make the unstructured data meaningful and we need to tag the data in same format of the structured data and apply the same techniques which are applied on the structured data. In the case of real time data some of the structured data may change to unstructured or semi-structured without notice.

\item Value

All data has hidden value, which must be revealed by employing range of technique's depending on the nature of value of the data. The technique's employed usually vary from the quantitative and investigative to discover the real value of the data ranging from expression of the consumers to financial status of the consumers.




\end{enumerate}


\section{Big Data Technology for Transportation}






\end{document}
