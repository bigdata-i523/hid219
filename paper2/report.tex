\documentclass[sigconf]{acmart}

\usepackage{graphicx}
\usepackage{hyperref}
\usepackage{todonotes}

%\usepackage{endfloat}
%\renewcommand{\efloatseparator}{\mbox{}} % no new page between figures

\usepackage{booktabs} % For formal tables

\settopmatter{printacmref=false} % Removes citation information below abstract
\renewcommand\footnotetextcopyrightpermission[1]{} % removes footnote with conference information in first column

\pagestyle{plain} % removes running headers

\begin{document}

\title{Big Data and Sentiment Analysis}

\author{Syam Sundar Herle}
\orcid{HID219}
\affiliation{%
  \institution{Indiana University}
  \streetaddress{711 N Park Ave}
  \city{Bloomington} 
  \state{Indiana} 
  \postcode{47408}
}
\email{syampara@iu.edu}

\begin{abstract}

Now a days, author express their feeling in the form text and the process of mining or analysis of the sentiment or opinion from these texts are known as sentiment analysis or opinion mining. Due to the dawn of the internet and social media application, people are sharing and communicating among them through these social media applications which plays major role in building up network among peoples. Here we try to explore more about sentiment analysis and how the problems of sentiment analysis is solved when big data technologies are used.

\end{abstract}

\keywords{i523, HID219, Big Data,Sentiment Analysis, Opinion Mining}

\maketitle

\section{Introduction}

\section{Big Data}

\section{Sentiment Analysis}

\section{conclusion}
\end{document}